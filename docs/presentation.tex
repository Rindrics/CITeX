\documentclass[dvipdfmx]{beamer}
\usepackage{pxjahyper} % cf. http://d.hatena.ne.jp/zrbabbler/20120528/1338221936
%\usetheme{Warsaw}
%\usecolortheme{seagull}
\usetheme[progressbar=head]{metropolis}
\usefonttheme{professionalfonts}

\useoutertheme{metropolis}
\useinnertheme{metropolis}

%\usecolortheme{seahorse}

\begin{document}
\title{Beamer のテスト}
\author{Akira Hayashi}
\date{\today}



\frame{\titlepage}

\frame{\frametitle{目次}\tableofcontents}

\section{セクション 1}
\frame{\frametitle{セクションのタイトル}
セクション内の文章。
}
\subsection{サブセクション 1-1}
\frame{\frametitle{サブセクションのタイトル}
サブセクション内の文章。
}

\section{セクション2}
\subsection{番号なしリストの使い方}
\frame{\frametitle{いわし類}
\begin{itemize}
\item マイワシ
\item カタクチイワシ
\item ウルメイワシ
\end{itemize}}

\subsection{ポーズの使い方}
\frame{\frametitle{資源評価の手順}
\begin{enumerate}
\item 月別漁獲量の集計\pause
\item 月別体長組成の整理\pause
\item VPA\pause
\item etc
\end{enumerate}}

\subsection{表の使い方}
\frame{\frametitle{予報の検証}
\begin{tabular}{ccc}
\hline
\textbf{海域} & \textbf{予報} & \textbf{結果} \\
\hline
北薩・薩南 & 前年を上回る & 前年を下回った  \\
日向灘 & 前年並み & 前年を下回った \\
豊後水道西側 & 前年並み〜上回る & 前年を下回った \\
\hline
\end{tabular}
}

\section{セクション3}
\subsection{ブロックの使い方}

\frame{\frametitle{さば類}
  \begin{block}{マサバ}
    \textit{Scomber japonicus} うまい.
  \end{block}\pause

  \begin{exampleblock}{ゴマサバ}
    \textit{Scomber australasicus} うまい.
  \end{exampleblock}\pause

  \begin{alertblock}{タイセイヨウサバ}
    \textit{Scomber scombrus} うまい.
  \end{alertblock}\pause

  つまりみんなうまい
}

\begin{frame}[standout]
Thank you!
\end{frame}

\end{document}
