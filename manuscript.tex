% Created 2019-02-28 Thu 09:35
% Intended LaTeX compiler: pdflatex
\documentclass[11pt]{article}
\author{}
\date{}
\title{マイワシ太平洋系群等の漁況予報}
\begin{document}

\maketitle

\section{漁況の経過}
\subsection{資源状態}
マイワシ太平洋系群の資源量は、1980 年代は 1,000 万トン以上の高い水準で推移したが、
1990 年代に入って急減し、2002 年以降 2009 年まで 10 万トン前後の低い水準で推移した。
その後、2010 年~2014 年に比 較的良好な加入が続いたこと、および漁獲圧が低下したことにより資源量は増加し
、2014 年には 100 万ト ンを上回った。
その後も良好な加入が続いたことにより資源量はさらに増加して、2017 年は 320 万トンと 推定された。

2015 年級群(4 歳魚)は、平成 30 年度資源評価において、加入量が 437 億尾と推定されており、
最近 10 年(以下、近年)において高い値となっている。2018 年における 3 歳魚としての漁獲状況は、
好調であった前年を下回ったものの、高い豊度を示している。
2015 年級群の残存資源は 2014 年級群の同時期を上回ると考えられる。

2016 年級群(3 歳魚)は、加入量が 280 億尾と推定されており、
2015 年級群には及ばないものの高い値 となっている。
2018 年における 2 歳魚としての漁獲状況は、好調であった前年並であった。2016 年級群の
残存資源は 2015 年級群の同時期を下回るものの、高い水準にあると考えられる。

2017 年級群(2 歳魚)は、加入量が 309 億尾と推定されており、2016 年級群を上回る高い値となっている。
2018 年における 1 歳魚としての漁獲状況も前年を上回っており、高い豊度を示している。
2017 年級群の残存資源は 2016 年級群の同時期を上回ると考えられる。

2018 年級群(1 歳魚)に対応する 2018 年の産卵量は、1,373 兆粒(2018 年 5 月までの暫定値)と、
2017 年(531 兆粒)を大きく上回った。マシラスは、渥美外海~駿河湾において高い水準の漁獲が見られ、 多く出現している。
0 歳魚としての漁獲も、4 月以降、房総海域において、6 月以降、相模湾~三陸海域に おいて例年に比べ多く見られており、高い豊度を示している。
沖合域の調査では、5 月~6 月の移行域幼稚 魚調査(中央水研・北水研)、
6 月~7 月の北西太平洋北上期浮魚類資源調査(東北水研)、
9 月~10 月の北西太平洋秋季浮魚類資源調査(中央水研)のそれぞれにおいて、
前年を上回る高い漁獲が見られている。
これらの情報から、現時点では不確実であるが、2018 年級群の加入量水準は 2017 年級群を上回ると考えられる。

2019 年級群(0 歳魚)については現時点ではその水準を予測できない

\subsection{来遊量,漁期・漁場・魚体}
\subsubsection{北薩〜熊野灘(まき網、定置網)}
\paragraph{来遊量}
北薩、薩南海域では、4 月まで 1 歳魚(2018 年級群)、5 月以降は 0 歳魚(2019 年級群)が漁獲 の主体となる
。2018 年級群の漁獲量は、5 月の発生以降、低調に推移していることから、
引き続き漁獲さ れる今期も期待できない。よって、来遊量は低調だった前年並となる。
日向灘では、5 月まで 1 歳以上、6 月は 0 歳魚が漁獲の主体となる。
今期の 1 歳以上の漁獲動向と一致する前漁期の三重県・高知県・鹿児島県のまき網漁獲量の相乗平均は前年を下回っている。
一方で、前年同 期の日向灘における漁獲状況は平年(過去 5 年)と比較しても非常に低調であった。
以上を踏まえ、今期 の来遊量は低調であった前年並となる。
豊後水道南部では、3 月まで 1 歳魚、2 歳魚、4 月以降 0 歳魚が漁獲の主体となる。
2018 年級群が主体で あった前漁期のまき網漁獲量が前年を上回って推移しており、
今期も明け 1 歳魚として引き続き漁獲され ると期待される。よって、今期の来遊量は前年を上回る。
宿毛湾、土佐湾では、1 歳魚が漁獲の主体となる。
前漁期の 2018 年級群の漁獲量は前年を上回って推移しており、
今期も明け 1 歳魚として引き続き漁獲されると期待される。
よって、今期の来遊量は前年を上 回る。
紀伊水道外域西部では、0 歳魚が漁獲の主体となる。
現段階での来遊量の予測は困難であるが、近年の漁獲傾向から前年並となる。
紀伊水道外域東部では、1 歳魚が漁獲の主体となる。
定置網は 2016 年以降、1 そうまき網は 2017 年以降、低調に推移している。
一方で潮岬沖の黒潮流軸の離岸傾向が継続した場合、本海域のまき網漁場の海水温は低下し、
マイワシの滞留条件としては良いと考えられる。以上を踏まえ、来遊量は前年並となる。
熊野灘では、1 歳以上が漁獲の主体となる。
太平洋系群の資源量から 3 歳以上の来遊量は前年を上回ると期待されるが、
前年は房総周辺~伊豆諸島で大規模な産卵が見られ、熊野灘への 3 歳以上の来遊が少なかった。
黒潮の流路は A 型で推移する予測となっており、今期も同様の状況になる可能性がある。
また、近年では、冬季に沿岸加入群の 1 歳魚、2 歳魚による産卵回遊が見られていたが、
2017 年、2018 年の秋季は いずれも 0 歳魚が不漁であったことから、
これらの来遊量は前年を下回ると予測される。以上を踏まえ、全体としての来遊量は前年並~下回る。
\paragraph{漁期}
各海域とも期を通じて漁獲される。
\paragraph{魚体}
北薩~豊後水道南部では、期前半は15 cm~20 cm前後の1歳以上、
期後半は5 cm~12 cm前後の0 歳魚が主体となる。
宿毛湾~紀伊水道外域では 11 cm~19 cm の 1 歳魚主体に、
17 cm 以上の 2 歳以上および、15 cm未満の0歳魚も漁獲される。
熊野灘では11 cm~19 cmの1歳魚と17 cm以上の2歳以上が漁 獲される。


\subsubsection{伊勢・三河湾〜相模湾(まき網、定置網、船曳網)}
\paragraph{来遊量}
伊勢・三河湾、渥美外海では、6 月以降 0 歳魚(2019 年級群)が主体となる。
黒潮の流路は A 型で推移する予測となっており、年明け以降も本海域沿岸の水温は高めで推移し、
春季の栄養環境は良好 でプランクトンも豊富になると考えられる。
11 月下旬時点ですでにマシラスの混獲も始まっており、来遊 や生き残り、成長に有利な海況となることが期待される。
以上より、今期の来遊量は好調であった前年並 となる。
駿河湾、相模湾西部では、期前半は 1 歳魚、2 歳魚(2018 年級群、2017 年級群)が主体となり、期後半 は 0 歳魚が主体となる。
2017 年級群は、マシラスとして非常に多く漁獲されており高い豊度を示している。
2018 年級群は、前漁期において 0 歳魚として多く漁獲されており、今期も引き続き 1 歳魚として漁獲され ると期待される。
以上より、今期の来遊量は前年を上回る。
相模湾東部では、1 歳魚が漁獲の主体となる。
夏以降、相模湾を含む太平洋側各海域で、2018 年級群の 漁獲が好調に推移しており、
引き続き 1 歳魚として漁獲されると期待される。
一方で、前年には大羽イワ シの来遊があり漁獲量が増加したが、
2 歳魚、3 歳魚の相模湾への来遊は海況次第であり、その予測は難しい。
以上を踏まえ、全体としての来遊量は前年並となる。
\paragraph{漁期}
伊勢・三河湾、渥美外海では 6 月以降、0 歳魚が漁獲される。
駿河湾、相模湾では期を通じて漁獲 される。
\paragraph{魚体}
伊勢・三河湾、渥美外海では 10 cm 以下の 0 歳魚が主体となる。
駿河湾、相模湾西部では、期前 半は18 cm以上の2歳魚と15 cm~16 cmの1歳魚が、
期後半は10 cm以下の0歳魚が主体となる。
相模湾 東部では13 cm~15 cmの1歳魚が主体となる。
\subsubsection{房総〜三陸海域、道東海域(まき網、定置網)}
\paragraph{来遊量}
各年級群の資源状態と近年の漁獲状況から、今期の漁獲対象は 1 歳魚(2018 年級群)、2 歳魚(2017
年級群)、3 歳魚(2016 年級群)および 4 歳魚(2015 年級群)となる。
2015 年級群は、近年において高い 加入量と推定されており、来遊量は好調であった前年を上回る。
2016 年級群は、高水準であるものの 2015 年級群を下回る加入量と推定されており、好調であった前年を下回る。
2017 年級群は、2016 年級群を上回 る高い加入量と推定されており、来遊量は前年を上回る。
2018 年級群は、調査船調査の結果や漁獲状況な どから高い加入量水準と期待され、現時点では不確実ではあるが、来遊量は前年を上回る。
以上から、全 体としての来遊量は前年を上回る。
\paragraph{漁期・漁場}
まき網の漁場は、1 月は房総海域~常磐海域、2 月~5 月は房総海域~鹿島灘、6 月は房総 海域~三陸南部海域で形成される。
定置網は、仙台湾~三陸南部海域において 2 月まで、および 5 月以降 に入網がみられる。
\paragraph{魚体}
1歳魚は12 cm~17 cm前後、2歳魚は16 cm~18 cm前後、3歳魚は17 cm~19 cm前後、4歳以上 は19 cm以上。
まき網では、期を通じて1歳~4歳魚が漁獲される。
定置網では、1歳魚主体に、2歳魚、 3 歳魚が混じる。

\end{document}
